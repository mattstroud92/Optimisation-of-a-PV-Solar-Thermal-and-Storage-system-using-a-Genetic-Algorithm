\section{Introduction and Background}

This section details the background information to the problem and the justification for this research project. The background details the aims and objectives from the study, a literature review on PV and solar thermal systems as well as various optimisation methods available.

\subsection{Justification}

As fossil fuels continue to deplete and our environment continues to deteriorate, the generation of energy from renewable sources is becoming more important. Energy sources such as PV and solar thermal panels can be used together to provide both electricity and heat in unison with their own respective storage means. Many householders or companies could use their areal roof space to implement their own systems and thus, there lies an opportunity to find the ideal solution since the roof space for buildings is generally very limited in size. To simplify the process of finding the ideal combination for a specific building in a specific location, an optimisation algorithm can be used to optimise any given set of demand requirements and resource potential and generate a Pareto front of solutions.

\subsection{Literature Review}

The following section gives a detailed review of PV and solar thermal systems both as separate entities. A brief discussion of PVT systems is given. It is evident that many studies have been conducted on these systems as separate entities, but there is less in terms of a hybrid arrangement in the context of a residential or commercial rooftop space. Four optimisation methods are detailed, which are: mixed-integer linear programming (MILP), energy hub modelling, stochastic modelling and the chosen method for this paper, Genetic Algorithm (GA). A brief discussion of other optimisation methods available is also given. 

There are three main systems that can be utilised in terms of energy 
generation from a solar resource. These are photovoltaic (PV), 
solar thermal and photovoltaic-thermal (PVT) systems.
PVT systems are another possible solution to the optimisation problem 
as they are capable of utilising the solar resource in a more efficient 
manner in comparison to separate PV and solar thermal systems \cite{HE20113369}
This study conducted at The University of Science and Technology of China 
also states that the PVT system has a lower electrical and thermal efficiency 
than that of PV and solar thermal as separate units. For this study, 
PVT systems will not be modelled. This study will look at creating an 
optimal hybrid system of PV and solar thermal systems as separate units 
within the same system.

\subsubsection{PhotoVoltaic and Solar-Thermal Systems}

Photovoltaic (PV) systems are the most commonly used system for both 
small-scale and large-scale energy generation projects. They work through 
utilisation of the photovoltaic effect, in which voltage is generated within 
the solar cell \cite{PVED}.

Studies by Bai et al \cite{BAI201541}and Diaz-Dorado et al \cite{Diaz} indicated that partial 
shading will have a detrimental effect on the module efficiency. This work 
was continued by others where a model developed at The University of Osijek 
in Croatia looked at finding the optimal PV configuration for a given areal 
space \cite{TOPIC2017750}. It looked at the number of rows and module angle as well as accounting 
for shading between rows of modules. The model used net present value (NPV) 
as the deciding factor for the most optimal outcome. The results indicated a 
high influence of shading between rows on the NPV output. The model, however, 
only looked at 5-degree increments for the module angle and as such, there is scope 
for a study with 1-degree resolutions. It only looked at one type of PV 
module and did not account for different module variants. Data was collected 
for the optimal tilt angle for different places in the Northern hemisphere 
for numerous different time periods \cite{CHANG20091274}. This data indicated that the tilt 
angle can be estimated from the duration of sunshine at a location. The work 
also indicates that cloud cover will have a negative effect on the production 
from PV systems. This work indicates that the place a module is installed on 
the earth’s surface will change the tilt angle required for optimal 
performance. Cloud cover, which can be more prevalent in certain geographic 
locations, will also largely affect how the optimal layout will be determined.

Solar thermal systems are another method of harvesting solar energy. 
For the creation of hot water, they work by absorbing the energy from the sun 
in the collector part of the module, which in turn heats up the fluid 
contained within the panels. This fluid is then used to heat the water in 
storage via a heat exchanger. There are two types of system that will be covered in this review, 
flat plate collectors and evacuated tube collectors. These are the two most 
suitable for a residential application.

A road-map developed by \cite{MAURER2015704}
for the future of building integrated solar
thermal systems states that spatial distribution, transmittance, absorptance 
and heat transfer are the main considerations for developing an efficient 
solar thermal system. They indicate however, that shading as a result of poor 
spatial distribution is less likely to occur on a roof-top and would be more 
of a consideration on a facade. These parameters will all contribute towards 
the performance of the collector system. \cite{BONHOTE2009799} discusses the performance of 
unglazed flat-plate collectors and concludes that performance will be reduced 
for space-heating purposes but can still perform adequately for domestic hot 
water provisions. The results do however indicate that wind shelter is 
necessary in order to gain an energy output that is adequate. These results 
indicate that there is a trade-off in performance for the cheaper price of 
the unglazed modules. The thermal performance of evacuated tube collectors 
with different tube shapes was discussed by \cite{KIM2007772} which indicated that when 
considering the effect of shadows on the tubes, a U-tube shape on a copper 
plate returned the best energy output. If the effect from shadowing is not 
considered, other models give better results and as such it can be said that 
considering the effect of the shadow cast from tube to tube is extremely 
important in order to obtain accurate results.

\subsubsection{Electricity and Thermal Storage Systems}

Short-term storage can play a vital role in providing energy during hours in which there is no sun. This can include during the night or during days in which there is a high level of cloud cover. For long-term storage, during the winter when there is less 
sunlight and increased cloud cover, it is important to ensure that in the 
case of domestic hot water and space heating that there is some form of 
storage during this period.

Storage of electricity from PV may come in the form of batteries or fuel 
cells. The use of batteries is the most common method of storage for PV \cite{StoragePVED}. 
A study conducted by \cite{DOUGLAS2016104} in which a dynamic model was developed for battery 
and hydrogen fuel cells within a PV context. The results indicated that 
battery storage was capable of sustaining provisions of low power loads for a 
longer period of time than that of the hydrogen fuel cell. The hydrogen fuel 
cell however, was capable of providing for higher power loads. They also 
discussed the fact that a hydrogen fuel cell would not require replacement, 
whereas the battery would. A comparison of hydrogen and battery storage was 
conducted by \cite{ZHANG2017397} where a positive scenario indicates that hydrogen storage 
achieved a higher NPV. However, in the more negative scenario, the battery 
storage is still the more economical option. A comparison of the two 
technologies in \cite{BELMONTE201621427} concluded that battery storage, due to its more mature 
status, is still the most economical option.  They indicated that the cost of 
the equipment required to produce the hydrogen is too expensive. It did 
however, discuss the environmental benefits of using hydrogen fuel over 
lithium-ion batteries.

Solar thermal energy storage can be achieved through various avenues. 
These include water storage tanks and ground-based storage as well as phase 
change heat batteries. For long term storage of thermal energy, \cite{PINEL20113341} 
indicates that the first account of seasonal storage was in 1939, where a 
tank was buried underground. Storage temperatures peaked at 90{\degree}C and dropped 
to 55{\degree}C over a 6-month time frame. A project in Canada utilised domestic hot 
water storage and a space heating storage tank which met 93\% and 64\% of the 
demands respectively \cite{cmhc}. A simulation conducted by \cite{Hugo} indicated that it 
is not economically feasible to have solar energy provide 100\% solar fraction. 
This was for a house in Canada where there is a colder climate and no 
governmental subsidies available at the time. 

Phase change materials (PCMs) are another means of long term thermal energy 
storage. PCMs work by utilising latent heat. The advantage of PCMs over 
traditional thermal energy storage is potentially higher thermal storage 
density and lower heating losses \cite{ALABIDI20125802}. The temperature at which the phase change occurs is important as it will determine the heat transfer 
process.

\subsubsection{Optimisation Methods}

In order to implement a hybrid PV and solar thermal system, it will be 
required to optimise it. This optimisation process can take into account 
various factors but will focus on one area such as economical or technical 
specification. mixed-integer linear programming, energy 
hub modelling stochastic simulation and genetic algorithms are discussed here. A brief 
discussion of other, less commonly utilised methods, is also given.

MILP is one method that can be used in order to optimise a system. 
It is typically performed in three main steps \cite{Colesmith}: 
\begin{itemize}
  \item Definition of a set of variables representing choices for optimisation. 
  \item Definition of problem constraints for the model.
  \item Statement of the objective function.
\end{itemize}

The MILP method has been used more commonly in solar thermal collectors and 
thermal storage chambers and not for a combination of PV, solar thermal and 
storage. A model developed by \cite{COSTA2014921} for combined heat and power (CHP) in a 
hospital accounted for thermal storage but did not account for storage losses, 
rather only the accumulation of storage over time. A CHP system with cooling 
is optimised in terms of its operation using MILP by \cite{BISCHI201412} over a short-term 
period. The energy hub approach is used by \cite{OREHOUNIG2015277} for a decentralised energy 
system in a neighbourhood in which thermal losses from storage are accounted 
for. This approach assumes the losses are a given percentage of the storage 
level. 

More recently, a general model for energy supply systems optimisation was 
developed by \cite{ITURRIAGA2017954} in which the MILP method was used. They used the method for 
a case study on a house in Bilbao, Spain, in which roof availability was a 
constraining factor. It contains a more detailed approach of the components 
within the system than other MILP models detailed earlier. \cite{OMU2016313} developed an 
iterative MILP model for the design and sizing of a solar thermal and storage 
system. Results indicated similar trends and values to that of EnergyPlus 
software and more accurate than a conventional MILP process. It works by 
iterating the solar collector area and thermal storage volume until an 
optimal solution is achieved. They indicated that future analysis should be 
performed on developing an optimal solution to a hybrid system of PV, solar 
thermal and storage for a fixed areal space. It can be said that the MILP 
approach is linear in nature and therefore cannot be used for problems of a 
non-linear nature.

The energy hub model (EH) was developed by Geidl and Andersson \cite{Geidl1} and 
is defined as a mixed energy carrier power system that provides three main 
features: Input/output, conversion and storage. This approach to hybrid 
energy systems presented a general mathematical model for power conversion 
and allowed for thermal and electrical systems to be optimised as part of 
one investigation \cite{Geidl1}. The initial model developed did not include storage 
as part of the optimisation. They continued their work in \cite{Geidl2} in which an 
objective function was given to minimise the cost of energy based on a 
system of different energy inputs and specific demand-side requirements. 
Several formulations for an energy hub model were evaluated by \cite{EVINS2014387}. These 
formulations used MILP as part of their expression to balance the demand 
and supply for the various forms of energy used. The new formulations 
extended the original EH model to account for system efficiency and storage 
losses. The addition of these further constraints resulted in a noteworthy 
influence on study results. This research did however, only look at the 
minimisation of carbon emissions.

The Monte-Carlo method involves randomly assigning values for each loop of 
a simulation. These values are recorded and once the simulation has been 
run for a certain number of loops, a distribution of the results can be 
given \cite{prob}. This distribution of results can then be used to classify which 
simulations output the most optimised system. \cite{CABRAL20101628} used a stochastic 
simulation to size a PV system and batteries in which results indicated 
more reliable results than with a deterministic approach. For this 
stochastic approach, historical data was used and is required for an 
accurate analysis. Similar work by \cite{TAN20105082} used Monte-Carlo simulation for 
battery sizing in a PV system. This simulation took into account load 
profiles, weather information, and local historical data. Sensitivity 
analysis was also conducted to test the robustness of the model to change. 
The results indicated that the overall investment of the system was 
influenced heavily by PV cost and export costs. The use of sensitivity 
analysis within a stochastic model would seem easier to implement than 
with a GA or MILP model.

A genetic algorithm (GA) is one method that can be used to optimise the 
system \cite{Goldberg}. It is based on natural selection. It revises a population of 
separate solutions. During each step of the process, the GA selects single 
entities within the population at random to be parents and uses them to 
generate the next set of solutions. Over successive generations, the 
population "evolves" to the optimised solution. The genetic algorithm uses 
three main types of rules at each step to create the next generation \cite{matlab}:
\begin{itemize}
  \item Selection (the separate solutions that will create the next set of potential solutions).
  \item Crossover (Combination of two solutions).
  \item Mutation (random changes to solutions).
\end{itemize}

This method is used by \cite{FREITAS2015562} for a PV system in an attempt to 
maximise the energy output and minimise the costs. Results indicate that this 
method is capable of finding an ideal solution from two case studies. It 
also states that it works better in smaller scale PV strings, which would be 
ideal for a rooftop scenario in which the space is limited. Similar work is 
conducted by \cite{kornelakis} in which the net economic benefit is maximised over a 
project lifetime for PV modules in a grid connected system. This study does 
not account for system losses or pitched rooftops. In this case, an 
evolutionary algorithm is utilised \cite{Back}. In both of these cases, PV modules 
are studied without the inclusion of solar thermal and storage. However, 
the algorithms utilised should be fully transferrable to include solar 
thermal and storage technologies. 

GA was used for combined cooling, heating and power (CCHP) which obtained an 
optimised system by following a general method. \cite{WANG20101325}. The GA follows a system 
that takes the input data and performs a ‘fitness’ calculation. Simply, this 
process is repeated until the GA parameters and objectives are satisfied. 
The results from the study indicated that the GA method was effective in 
optimising the system. A review of different optimisation methods conducted 
in Istanbul by \cite{ERDINC20121412} gives an excellent summarising figure and description of the GA approach in regard to sizing energy systems. The 4 input categories of meteorological, economical, optimisation criteria and fitness function are fed into the GA procedure. This review indicates that the largest advantage of using GA is that it is a very efficient method for finding the optimum solution but does indicate that GA is a difficult process to implement due to its complexity.

A hybrid optimisation model was developed using genetic algorithms by \cite{DUFOLOPEZ200533}. The model, named HOGA (Hybrid Optimisation by Genetic Algorithms) was capable of designing a PV-diesel hybrid system. It outputs the quantity of PV panels 
and number of batteries required alongside the diesel requirements for the 
optimised solution in terms of economics. It does not account for uncertainty 
in solar irradiation data however, and this is something that can be 
implemented into a new model.

There are many other optimisation methods available outwith the four 
that have been discussed above. Many of these have not been used in 
the context of renewable energy system optimisation.  A review of 
optimisation methods indicated particle swarm optimisation, ant 
colony optimisation, artificial immune system, game theory and honey 
bee mating algorithm are all potential future methods that can be used 
for sizing and optimising of hybrid renewable energy systems \cite{ERDINC20121412} . 

\subsection{Aim and Objectives}

The aim of the work is to optimise the performance of PV and solar thermal hybrid systems for a building. In order to achieve this aim, the following objectives have been laid out: 
\begin{itemize}

\item Collect and analyse literature for PV and solar thermal systems. 
\item Collect and analyse storage technologies for both electricity and heating demands. 
\item Obtain resource and demand data sets for the project. 
\item Develop a methodology consisting of equations for minimising objective functions.
\item Optimise a system of PV, flat-plate collectors, Li-Ion batteries and PCM heat batteries using a genetic algorithm.
\item Perform a critical analysis of the results obtained.
\end{itemize}



